% Created 2022-10-25 Tue 19:05
% Intended LaTeX compiler: pdflatex
\documentclass[11pt]{scrartcl}
\usepackage[utf8]{inputenc}
\usepackage[T1]{fontenc}
\usepackage{graphicx}
\usepackage{longtable}
\usepackage{wrapfig}
\usepackage{rotating}
\usepackage[normalem]{ulem}
\usepackage{amsmath}
\usepackage{amssymb}
\usepackage{capt-of}
\usepackage{hyperref}
\usepackage{minted}
\usepackage{bm}
\usepackage{siunitx}
\usepackage{natbib}
\bibliographystyle{unsrtnat}
\usepackage{siunitx}
\date{\today}
\title{Precipitation retrieval using convolutional neural networks for METOP and METEOSAT platforms\\\medskip
\large Associated scientist activity \\ Final report}
\hypersetup{
 pdfauthor={},
 pdftitle={Precipitation retrieval using convolutional neural networks for METOP and METEOSAT platforms},
 pdfkeywords={},
 pdfsubject={},
 pdfcreator={Emacs 28.1 (Org mode 9.5.2)}, 
 pdflang={English}}
\begin{document}

\maketitle


\section{Introduction}
\label{sec:org8b2f08a}

This report presents the results of the associated scientist activity \texttt{AS21-05}. 


\subsection{Background}
\label{sec:orgc337a7d}


Real-time precipitation monitoring forms the basis for various meteorological
and hydrological applications. Ground-based precipitation radars and satellites
are the only observation systems suitable for large-scale, real-time
precipitation monitoring. Although ground-based radar networks provide the most
reliable precipitation measurements, space-borne observations can complement
ground-based observations where the former may be unavailable or otherwise
corrupted.


Today, a wide range of space-borne sensors provide observations that can be used
to monitor precipitation. However, the sensor characteristics strongly influence
the quality and availability of the resulting precipitation estimates. The
principal factors determining the suitability of a sensor for precipitation
monitoring are its observation frequency and the orbit into which the sensor is
placed.


Visible (VIS) and infrared (IR) observations are primarily sensitive to the
upper parts of clouds. Although the observed cloud structures can provide
limited information on the presence of precipitation, VIS and IR observations do not
provide a direct signal from precipitation near the surface. On the other
hand, microwave (MW) observations are directly sensitive to the emission and
scattering signatures of precipitation. However, the relationship between
observations and the corresponding precipitation depends on the precipitation
microphysics and, for passive observations, the surface properties and the
thermodynamic structure of the atmosphere. This leads to significant, inherent
retrieval uncertainties even for microwave-based measurements.

The most common orbits for precipitation retrievals are geostationary and
low-earth orbits. The principal advantage of geostationary orbits is that they
provide observations at high temporal resolution  in
Europe) for all locations within their field of view. The disadvantage of
geostationary platforms is their large distance from Earth, which makes them
unsuitable for MW sensors due to their limited spatial resolution. The large
distance from Earth is less of an issue for VIS and IR sensors, which can
achieve resolutions of a few kilometers and less even from geostationary orbits.
However, the resolution of geostationary observations decreases with
increasing latitude limiting their use for precipitation retrievals to regions within
\(\SI{-70}{\degree N}\) and \(\SI{70}{\degree N}\).

Platforms in low-Earth orbit are suitable for VIS, IR, and MW sensors. Their
proximity to the surface allows them to achieve higher spatial resolutions than
from a geostationary orbit. Nonetheless, the resolution of microwave sensors in
low-earth orbits is still lower than that of most current geostationary VIS and IR
observations. Compared to geostationary sensors, the principal disadvantage of
sensors in low-earth orbit is their poorer temporal coverage caused by their
limited swath width. However, the temporal resolution increases with increasing
latitude for  orbits passing over the poles. The increased
availability of MW observations over high latitudes adds to their value for
precipitation retrievals in these regions.


Because of the complementary characteristics of different sensors and orbits,
meteorological agencies operate multiple platforms that combine different orbit
and sensor types. For example, EUMETSAT operates the METEOSAT and METOP programs
to ensure the availability of VIS, IR and MW observations for operational
meteorological use. The currently operational METEOSAT second generation (SG)
platforms carry the Spinning Enhanced Visible and Infrared Imager (SEVIRI,
\citeauthor{aminou02}, \citeyear{aminou02}), which provides VIS and IR
observations from a geostationary orbit every 15 minutes over Europe. Additionally, the
currently operational METOP satellites carry, among others, the Microwave
Humidity Sounder (MHS, \citeauthor{bonsignori07}, \citeyear{bonsignori07}) and
the Advanced Very High Resolution Radiometer (AVHRR) providing MW and VIS and IR
observations from a low-earth orbit. While many more sensors provide
meteorologically valuable observations from space, this study focuses on those
named above.


\subsection{Motivation}
\label{sec:orga93baac}


The motivation for this study stems from the observation that current
precipitation retrievals underutilize available satellite imagery.

For one, most retrieval algorithms process pixels in the observations in
isolation and thus neglect the structural information in the satellite
observations. However, as recent advances in retrieval methodology
\citep{pfreundschuh22} have shown, exploiting the spatial information in
satellite observations can lead to notable improvements in retrieval accuracy.

Moreover, current, operational NRT precipitation retrievals are typically
designed to process observation from a specific sensor type. They thus tend to
neglect a significant part of the available observations and directly inherit
the limitations of the underlying sensor in terms of sensitivity to
precipitation as well as temporal and spatial resolution.

\subsection{Overview}
\label{sec:org073dc78}


We have developed a prototype for a machine-learning-based NRT precipitation
retrieval. Similar to other state-of-the-art precipitation retrievals
\citep{pfreundschuh22, pfreundschuh22a}, the implementation is based on a
convolutional neural network (CNN) allowing the retrieval to make use of
structural information in the satellite imagery. Moreover, the retrieval
combines observations from different sensors and can merge observations in time.
In terms of functionality, the closest algorithm in the published literature is
probably the work by \citet{gorooh22}, which combines microwave and
geostationary IR observations for retrieving precipitation. However, their
algorithm only works where both observations are available and thus inherits the
coverage limitations of the MW observations used in the retrieval. Moreover, the
retrieval does not combine observations in time. Our algorithm was designed to
overcome these limitations in order to leverage the full potential of currently
available satellite imagery for real-time precipitation monitoring.

A limitation of this study is that radar reflectivities are used as a proxy for
precipitation rates. The motivation for this is that the Baltrad measurements
use a simple power-law relationship to derive rain rates from reflectivities and
do not distinguish precipitation by type or phase. Moreover, the reflectivity
fields exhibit visible artifacts even after quality control. Therefore, we do
not expect the measurements to provide accurate, quantitative precipitation
estimates and thus limit the retrieval algorithms' evaluation to retrieved radar
reflectivities, which contain the same information as the derived precipitation
estimates.

The implementation of the retrieval algorithm is described in
section~\ref{sec:method}. The retrieval performance is assessed in
section~\ref{sec:results}. Multiple retrieval configurations have been developed
and tested to assess the added value of combining observations from different
sensors and merging observations in time. In addition, we explore the
possibility of using the merged retrieval to predict the evolution of
precipitation fields into the future. The report closes with a brief
discussion of the results and suggestions for future development in
section~\ref{sec:conclusions}


\section{Algorithm description}
\label{sec:method}


\subsection{Training dataset}
\label{sec:org4dcfa76}

The training dataset consists of collocations of Baltrad ground-based
 radar measurements collocated with observations from SEVIRI, AHRR,
MHS, and ATMS. Training samples are extracted for minutes \(0, 15, 30, 45\) of
every hour. Observations from platforms in low-Earth orbit are mapped to the
closest quarter-hour and stored only when present. All data from the year 2020
are used for training.


Baltrad radar reflectivities are used as the training target to train the retrieval algorithms. Only pixels with a quality index larger than 0.8 are used during the training in order to minimize the risk of training the retrieval to reproduce corrupted Baltrad measurements. Unfortunately, the quality index does not take into account the distance from the radar, which may reduce the reliability of the reference data. However, given that this study focuses on the relative accuracy of the retrievals, we have considered this issue of minor importance and not pursued it further.

\subsection{Neural network model}
\label{sec:org623391a}

A precipitation retrieval that fully exploits the synergies between different
satellite observations poses two technical challenges that the underlying
implementation must address.
\begin{enumerate}
\item Since observations from different platforms are available at different
  times, the neural network must allow for the merging of observations with
  irregular availability. In particular, it is essential for operational use
  that the retrieval produces accurate results even when some observations are
  missing.

\item Observations from platforms in low-earth orbits are available at discrete
  times. However, due to the continuous evolution of the atmospheric state, the
  observations can be expected provide useful information even after the immediate
  overpass. Therefore, to fully leverage the potential of available satellite
  observations, the retrieval should be able to integrate observations from
  previous time steps.
\end{enumerate}


The c neural network architecture that has been developed to address these two
challenges is displayed in Fig.~\ref{fig:arch}. The network consists of two
parallel branches, each comprising a U-Net-type \citep{ronneberger15} encoder
and decoder(, whose stages are represented by the gray pyramid stumps in
Fig.~\ref{fig:arch}). The observation branch ingests the satellite observations
and transforms them into a multi-scale representation of the atmospheric state,
the \textit{hidden state}, while the time-propagation branch transforms the
hidden state from the previous time step.


\subsubsection{Basic blocks}

The neural network model uses convolution blocks from the ConvNext \citep{liu22}
model as basic building blocks. Several of these blocks are combined to form the
stages of the encoder and decoder in the observation and time-propagation
branches. Additionally, these blocks are used in the stems of the network and
the merging of different data streams.

\begin{figure}
\begin{center}
  \includegraphics[width=\linewidth]{arch.png}
  \caption{
    The neural-network architecture used for the merged precipitation
    retrievals. The network comprises two separate branches: The observation
    branch, which encodes the satellite-observation input into a the network's
    hidden state, and a time-propagation branch, which transforms the
    hidden state from the previous time step. The outputs from the
    time-propagation and observation branch are merged using dedicated merge
    blocks (M) to form the final hidden state for the current time step. A
    fully-connected head (H) then transform the hidden state into reflectivity
    estimates.
  }
  \label{fig:arch}
\end{center}
\end{figure}

\subsubsection{Observation branch}

The observation branch ingests the satellite observations. The observations are
grouped according to their resolution yielding three inputs types: AVHRR
observations at the base resolution of $\SI{2}{\kilo \meter}$, the geostationary
observations at a resolution of $\SI{4}{\kilo \meter}$ and the microwave
observation at a resolution of $\SI{8}{\kilo \meter}$.

Each input is first fed into a corresponding stem ($S_1, S_2, S_3$ in
Fig.~\ref{fig:arch}) consisting of a single ConvNext block. The inputs at
$\SI{4}{\kilo \meter}$ and $\SI{8}{\kilo \meter}$ resolution are then merged
with the corresponding features from the encoder using a single ConvNext block.
The merged features are then fed into the subsequent encoder stage.

The number of features is doubled with each stage of the encoder, starting with
16 at the highest resolution. The encoder consists of 5 stages, each containing
4 ConvNext blocks. Downsampling is performed using the same downsampling blocks as
the original ConvNext architecture.

The decoder stages consist of only one ConvNext block each. Upsampling is
performed using bi-linear interpolation. Although not shown in
Fig.~\ref{fig:arch}, skip connections are used to connect the outputs from the
encoder stages to the corresponding decoder stages. The hidden state is obtained
by projecting the output from each decoder stage to 16 channels using a simple
$1 \times 1$ convolution layer.

\subsubsection{Time-propagation branch}


 The encoder-decoder structure of the time-propagation branch is similar to that
 of the observation branch, except that the input is the hidden state from the
 previous time step. The output from the time-propagation branch is merged with
 the observation-based hidden state obtained from the observation branch. The
 merging is performed using a separate ConvNext block for each scale. The
 resulting merged hidden state is taken as the final hidden state corresponding
 to the current time step. The time-propagation branch is only present in
 retrievals that perform temporal merging of observations.


\subsubsection{Calculation of retrieval output}


Radar reflectivities are calculated from the hidden state using a separate
network head. The head consist of layers of $1 \times 1$ convolutions followed
by layer norm \citep{ba16} and GELU \citep{hendrycks16} activation functions. A
final $1 \times 1$ convolution transforms the output to 64 values per pixel,
which are interpreted as the quantiles of the posterior distribution of the
corresponding dBZ.

\subsection{Training}

The retrieval uses quantile regression \citep{pfreundschuh18} to produce
probabilistic estimates of the radar reflectivity. The network is trained to
predict the quantiles corresponding to quantile fractions $\tau = [\frac{1}{65},
  \ldots, \frac{64}{65}]$. The training is performed using the AdamW
\citep{loshchilov18} optimizer and a cosine-annealing learning-rate schedule
\citep{loshchilov16} with warm restarts every 20 epochs. The training is
restarted until the training loss has converged.

The training data uses all available radar measurements over the Baltrad domain
from 2020. Samples are generated by extracting random crops of a size of 256
pixels at $2-\si{\kilo \meter}$ resolution and transforming them using random
flip and transpose operations. For one epoch, one sample is extracted from every
available radar scene. For the training of the microwave-only retrieval,
the scenes were restricted to scenes for which microwave observations are
available.


\subsection{Retrieval configurations}

To assess the benefits of merging satellite observations from different sensors
and time steps, we have developed several retrieval configurations that
consecutively integrate more observations into the retrieval. The geostationary-
(\textit{Geo}) and microwave-only (\textit{MW}) configurations use only
observations from the geostationary sensor (SEVIRI) and the microwave sensors
(MHS, ATMS), respectively. They are used as baseline retrievals because they are
most representative of conventional precipitation retrieval algorithms. The
\textit{All} configuration uses all observations considered in this study, i. e.
observations from AVHRR, SEVIRI, MHS and ATMS. Finally, the \textit{All, merged}
configuration uses observations from all sensors and merges them temporally. The
configurations are summarized in table~\ref{tab:configurations}.

\begin{table}[hbpt]
  \begin{center}
  \begin{tabular}{l|cc}
    Name & Input observations & Temporal merging \\
    \hline
    Geo & SEVIRI & No \\
    MW & MHS, ATMS & No \\
    All & SEVIRI, AVHRR MHS, ATMS & No \\
    All, merged & SEVIRI, AVHRR, ATMS & Yes
  \end{tabular}
  \end{center}
  \caption{Retrieval configuration assessed in this study.}
  \label{tab:configurations}
  \end{table}


The configurations all share the same basic architecture illustrated in
Fig.~\ref{fig:arch}. The stems and parts of the encoder that are not required
are omitted for configurations that only use a subset of the satellite
observations. Similarly, the time-propagation branch is omitted for
configurations that do not merge observations in time.


\section{Results}
\label{sec:results}

The accuracy of the retrieval is assessed using all observations from the months
May and December 2021. The retrieval was run on the full Baltrad domain and the
results compared to corresponding the Baltrad measurements.

\subsection{Accuracy}


Bias, mean-squared error (MSE) and correlation coefficients for all
configurations are shown in Fig.~\ref{fig:metrics}. The evaluation considers
only pixels at which SEVIRI, AVHRR and microwave observations are available to
ensure a fair comparison to ensure a just comparison.

The \textit{GEO} configuration exhibits the largest biases, which are
significantly smaller for the \textit{MW}, \textit{All} and \textit{All,
  merged} configurations. In terms of MSE, \textit{Geo} has the largest errors,
followed by \textit{MW}, \textit{All}, and \textit{All, merged}. The decrease in
error for the \textit{All} configuration indicates that the retrieval
successfully learned to leverage complementary information in the satellite
observations, which improves the accuracy of the reflectivity retrieval. The temporal
merging of observations leads to further increases in retrieval accuracy. The
same tendencies are observed  for the correlation between retrieved and
reference reflectivities. Although the overall accuracy is lower for December,
the relative performance of the configurations is the same as for May.

\begin{figure}
  \centering
  \includegraphics[width=\textwidth]{metrics}
  \caption{
    Error metrics for the different retrieval configuration calculated for May (first row) and
    November 2021 (second row).
    }
  \label{fig:metrics}
\end{figure}

Scatter plots of the reference and retrieved radar reflectivities are shown in
Fig.~\ref{fig:scatter}. The scatter plots reveal clear differences between
\textit{Geo} and the other configurations. The \textit{Geo} configuration
exhibits the strongest tendency to underestimate high reflectivities. The
\textit{MW} configuration significantly improves the retrieval of high
reflectivities. Compared to \textit{MW}, the \textit{All} and \textit{All,
  merged} configurations principally improve the retrieval of moderate and high
reflectivities.


\begin{figure}
  \centering
  \includegraphics[width=\textwidth]{scatter}
  \caption{
    Scatter plots for the different retrieval configuration for May (first row) and
    November 2021 (second row).
  }
  \label{fig:scatter}
\end{figure}

\subsection{Case studies}

A case study of consecutive retrievals from 20 May, 2021 is displayed in
Fig.~\ref{fig:case_1_seq}. All retrievals reproduce the large-scale structure of
the reflectivity field fairly well but underestimate the highest reflectivities.
This tendency is most pronounced for the \textit{Geo} configuration, which
retrieves the lowest reflectivities for the precipitation system that extends
north from the Gulf of Bothnia. The configurations that make use of micorwave
observations (\textit{MW}, \textit{All}, \textit{All, merged}) are more
successful in reproducing the magnitude of the reflectivities. Comparing the
\textit{MW}, \textit{All}, \textit{All, merged} configuration, the results show
a clear improvement in the spatial resolution  when VIS and IR
observations are incorporated into the retrieval.

The final row of Fig.~\ref{fig:case_1_seq} also shows the retrieved
precipitation probability from the discontinued legacy retrieval. The retrieval
outputs probabilities for light, moderate and intense precipitation. They have
been combined to a general precipitation probability by taking the maximum of
the three retrieved probabilities. The retrieval produces large regions of
low precipitation probabilities, which strongly overestimate the real extent
of the radar echoes, while the regions with high precipitation probability
strongly underestimate the extent of the observed precipitation.


\begin{figure}
  \centering
  \includegraphics[width=\textwidth]{case_1_seq}
  \caption{
    Retrievals for six consecutive SEVIRI observations from 20 May, 2021
    starting at 7:30 UTC. The first row shows the Baltrad reference
    measurements. The second, third, fourth and fifth rows show the results from
    the \textit{MW}, \textit{Geo}, \textit{All} and \textit{All, merged}
    retrieval configurations, respectively. The sixth row shows a map of the
    observation availability. The seventh row shows the probability of
    precipitation derived from the discontinued legacy precipitation retrieval.
  }
  \label{fig:case_1_seq}
\end{figure}

A second case study from 14 December, 2021 is shown in Fig.~\ref{fig:case_2_seq}.
Two large precipitation systems can be seen over the west-coast of Norway, as well
as some precipitation in the east of the domain. Again, \textit{Geo} underestimates
precipitation for the two precipitation systems over the Norwegian coast, which is
clearly improved for the retrievals that incorporate microwave observations.

A clear impact of the temporal merging can be observed here on the eastern flank
of the precipitation system impacting Southern Norway. At 20:30 both
\textit{All} and \textit{All, merged} reproduce the shape of the precipitation
cell that extends inlands. At 20:45, however, the shape of the cell is strongly
blurred in the \textit{All} results but still truthfully represent in the
\textit{All, merged} results.



\begin{figure}
  \centering
  \includegraphics[width=\textwidth]{case_2_seq}
  \caption{
    Retrievals for six consecutive SEVIRI observations from 14 December, 2021.
    The first row shows the Baltrad reference measurements. The second, third,
    fourth and fifth rows show the results from the \textit{MW}, \textit{Geo},
    \textit{All} and \textit{All, merged} retrieval configurations,
    respectively. The sixth row shows a map of the observation availability.
  }
  \label{fig:case_2_seq}
\end{figure}

\subsection{Precipitation forecasts}

The \textit{All, merged} configuration can also be used to predict the temporal evolution of the reflectivity fields. To explore the algorithm's potential for precipitation forecasting, we have performed forecasts for the first ten days of May 2021. For this, the retrieval has been trained on sequences of 32 consecutive observations, of which the first 16 contain observations but the last 16 do not.

 Fig.~\ref{fig:forecast} shows the mean-squared error and correlation of the
 forecasts compared to the reference reflectivity fields. The corresponding
 results of a persistence forecast using the retrieval results at $t =
 \SI{0}{\minute}$ are also shown. The retrieval clearly outperforms the
 persistence forecast indicating that it has some skill for precipitation
 forecasting. Unfortunately, we were not able to evaluate the quality of the
 forecast against the currently operational precipitation nowcasting algorithm
 within this project.


\begin{figure}
  \centering
  \includegraphics[width=\textwidth]{forecast}
  \caption{
    Accuracy metrics for precipitation forecasts performed with the \textit{All, merged}
    configuration.
  }
  \label{fig:forecast}
\end{figure}

A case study of a precipitation forecast is presented in
Fig.~\ref{fig:case_forecast}. The figure shows the reference reflectivity fields
together with the mean of the predicted reflectivities and the predicted
probability that  reflectivities higher than $\SI{0}{\deci \bel}$ are observed in
each pixel. Due to the probabilistic nature of the forecast, the predicted mean
reflectivity should be interpreted not as a single realization of a forecast but
rather as the mean of an ensemble of forecasts. This approach has the advantage that
statistical inference, such as the calculation of exceedance probabilities, can
be performed from just a single forecast.


Overall, the forecast captures the evolution of the principal precipitation
features, such as the eastern flank of the frontal system over the baltic sea.
Also, the precipitation system propagating in the western direction over
northern Finland is represented correctly in the forecast. However, the forecast
misses the smaller-scale precipitation developing in the South-East of the
scene. Moreover, the evolution of the precipitation system over northern Finland
into an eight-shaped structure seems to indicate that the forecast reproduces
radar artifacts in its predictions. Interestingly, this structure can be
partially observed even in the reference measurements, indicating that the
forecast correctly predicts what it was trained to predict.


\begin{figure}
  \centering
  \includegraphics[width=\textwidth]{forecast_case}
  \caption{
    Four prediction steps from a reflectivity forecast initialized
    at 8:00 on 6 May 2021. The first row of panels show the reference
    measurements corresponding to the forecasts. The second row
    of panels show the posterior mean of the probabilistic reflectivity
    forecast. The third row of panels shows the predicted probability
    of the reflectivity exceeding $\SI{0}{\deci \bel}$.
  }
  \label{fig:case_forecast}
\end{figure}

\section{Summary and conclusions}
\label{sec:conclusions}

This study presents a neural-network-based NRT precipitation retrieval algorithm that combines observations from multiple satellite sensors and merges them in time. The evaluation of the algorithm shows that the retrieval can leverage synergies between observations from different sensors and successive time steps.

While there exist certain retrieval algorithms \citep{gorooh22} that combine observations from multiple sensors, they typically only work for the case where all observations are available. The algorithm developed in this study works with any subset of observations, which is crucial for achieving high spatial coverage and temporal resolution in an operational setting.

Other notable precipitation retrieval algorithms that use temporal information
are IMERG \citep{huffman20} and CMORPH \citep{joyce11}. These algorithms use
motion fields derived from geostationary satellites to interpolate between
retrievals from microwave sensors. However, these techniques only aim to
counteract the absence of observations and  cannot improve the accuracy of
the original microwave retrievals. The retrieval presented here improves the
accuracy even when observations from all considered sensors are already
available. This shows that observations are combined synergistically, not only
between sensors, but even across time steps. To the best of our knowledge, this
is the first precipitation retrieval algorithm that achieves this.

Finally, we have demonstrated that the merged retrieval can predict the temporal
evolution of precipitation fields. Although the forecasts presented here are
merely a proof of concept, this is an exciting direction for future research.

The retrieval presented in this study constitutes a first step toward the next generation of data-driven precipitation retrievals. Our results demonstrate the benefits of the machine-learning-based approach, which allows combining observations from various sensors and successive time steps synergistically, which, so far, could only be achieved through data assimilation. However, due to the difficulty of representing hydrometeors, data assimilation is not well suited for precipitation retrievals.

Considering the limited extent of this project, we expect the results presented
here to be merely a glimpse of what is possible with the approach. The presented
retrieval can easily be extended to incorporate observations from additional
sensors. Moreover, only a fraction of the available training data has been used
in this project. Given the empirical evidence from other deep learning
applications, it can be expected that increasing the training data can
substantially improve the retrieval. Finally, future developments should also
revisit the neural-network architecture used in the retrieval and improve the
quality of the reference data.




\bibliography{references}

\end{document}
